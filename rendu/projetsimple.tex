%%%%%%%%%%%%%%%%%%%%%%%%%%%%%%%%%%%%%%%%%
% Short Sectioned Assignment
% LaTeX Template
% Version 1.0 (5/5/12)
%
% This template has been downloaded from:
% http://www.LaTeXTemplates.com
%
% Original author:
% Frits Wenneker (http://www.howtotex.com)
%
% License:
% CC BY-NC-SA 3.0 (http://creativecommons.org/licenses/by-nc-sa/3.0/)
%
%%%%%%%%%%%%%%%%%%%%%%%%%%%%%%%%%%%%%%%%%

%----------------------------------------------------------------------------------------
%	PACKAGES AND OTHER DOCUMENT CONFIGURATIONS
%----------------------------------------------------------------------------------------

\documentclass[paper=a4, fontsize=11pt]{report} % A4 paper and 11pt font size
\usepackage[margin=0.7in]{geometry}
\usepackage[T1]{fontenc} % Use 8-bit encoding that has 256 glyphs
%\usepackage{fourier} % Use the Adobe Utopia font for the document - comment this line to return to the LaTeX default
\usepackage[francais]{babel} % English language/hyphenation
\usepackage[utf8]{inputenc}
\usepackage[babel=true]{csquotes} 
\usepackage{amsmath,amsfonts,amsthm} % Math packages

\usepackage{lipsum} % Used for inserting dummy 'Lorem ipsum' text into the template


%----------------------------------------------------------------------------------------
%	TITLE SECTION
%----------------------------------------------------------------------------------------

\newcommand{\horrule}[1]{\rule{\linewidth}{#1}} % Create horizontal rule command with 1 argument of height

\title{	
\normalfont \normalsize 
\textsc{Master SAR} \\ [25pt] % Your university, school and/or department name(s)
\horrule{0.5pt} \\[0.4cm] % Thin top horizontal rule
\huge Projet ARA \\ % The assignment title
\horrule{2pt} \\[0.5cm] % Thick bottom horizontal rule
}

\author{Alexandra Hospital, Thomas Medioni} % Your name

\date{\normalsize\today} % Today's date or a custom date

\begin{document}

\maketitle % Print the title

%----------------------------------------------------------------------------------------
%	PROBLEM 1
%----------------------------------------------------------------------------------------

\section{Exercice 1}

\subsection{q1}

L'article traite des problèmes d'élection de leader au sein d'un système mobile ad hoc. En utilisant l'algorithme d'arbres couvrants de Dijkstra et Scholten dans les réseaux statiques et en l'adaptant aux problématiques des réseaux mobiles ad hoc, dans lequel l'état des composantes connexes n'est pas assuré dans le temps, les auteurs fournissent un algorithme permettant d'assurer à terme un leader unique par composante connexe. Le taux d'élection est supérieur à 95 pourcent en fonction de l'algorithme de transport.

\subsection{q2}

L'algorithme fait les hypothèses suivantes sur le système :

Il est modélisable par un graphe dynamique non orienté dont les noeuds sont mobiles et les arêtes représentent l'existence d'une connexion entre deux noeuds, c'est-à-dire la capacité pour un noeud à communiquer directement avec un autre.

Chaque noeud possède une valeur exprimant son aptitude à être leader ainsi qu'un identifiant unique et ordonné. La valeur du noeud la plus élevée représente le noeud le plus apte à être leader, l'identifiant unique sert à départager deux candidats à aptitudes égales.

Les canaux de communication entre les noeuds sont bidirectionnels et respectent un ordre FIFO entre deux noeuds adjacents.

La mobilité des noeuds peut impliquer des changements de topologie entraînant des partitions ou des fusions du réseau. De plus, les noeuds peuvent crasher de manière arbitraire .

Les messages ne sont délivrés que si l'émetteur et le récepteur sont connectés durant l'émission du message.

Les buffers de réception sont infiniment larges de manière à ce que il n'y ait jamais de dépassement de capacité.

Les communications s'effectuent par échange de messages point-à-point de manière asynchrone.

\subsection{q3}

Les élections concurrentes au sein d'une même composante connexe sont gérées grâce au couple de valeurs <num, id> selon la règle suivante :
Soient <num1, id1> et <num2, id2> deux indexes de noeuds concurrents. le noeud possédant le num le plus grand l'emporte.
Si num1 = num2, le noeud qui a le plus grand identifiant l'emporte.

C'est le consensus : tous les noeuds s'accordent sur un noeud qui lancera l'élection (qui sera la racine de l'arbre).


\subsection{q4}

Processus i :

Variables locales :
Ni // tableau de voisins
Si  // tableau de voisins dont on attend encore un message ack (NULL au debut)
parent // initialisé à null

Leader\_election() {
	if i == source
		send <election, i> to Ni	
}

upon reception <election, j>
	if parent == null
		parent = j
		Si = Ni \ parent
		send <election, i> to Si
		while (Si not empty)
			wait unti receive <ack, s>
			upon reception <ack, s>
			Si = Si \ s

	send <ack, i> to parent
			

upon reception <ack, j>
	Si = Si \ j

upon reception <leader, j>
	send <leader, i> to Ni \ parent


Complexité en messages : O(n)
Chaque noeud envoie un message Election à tous ses voisins sauf son parent et attend un ack d'eux.
Le noeud source "broadcast" la valeur du leader.

 
\subsection{q5}
C'est le système de heartbeat avec temporisation : si on ne reçoit pas de message "je suis en vie" d'un noeud depuis un certain temps borné, on considère qu'il est en panne. 

\subsection{q6}

Le modèle de mobilité Random Waypoint est utilisé pour simuler les changements de position, vitesse, et accéleration en fonction du temps. Ce modèle est fortement utilisé
pour étudier les comportements des noeuds au sein des MANET. Chaque noeud est d'abord immobile pendant un nombre fixé de secondes, puis choisit une destination aléatoire, ainsi qu'une vitesse aléatoire choisie entre une borne
minimum et maximum. Le noeud se déplace alors jusqu'à sa destination puis effectue une pause avant de choisir une nouvelle destination et une nouvelle vitesse de manière aléatoire. Ce comportement est observé par tous les noeuds jusqu'à 
la fin de la simulation.

\section{Exercice 2}
Dans cette section, nous allons expliquer nos choix d'implémentation des algorithmes de l'exerccice 2.

\subsection{q1}
Classe RandomWayPointProtocol.java \\
Chaque noeud va tirer aléatoirement sa position x et y de départ et de destination. 
Dans la fonction processEvent TODOOOOOOOOOOOOOOOOOOOOO

\subsection{q2}
Classe EmitterImpl.java\\
La méthode d'émission envoie un message à chaque noeud qui fait partie de ses voisins, c'est-à-dire qui est à une distance inférieure à scope.

\subsection{q3}
Classe ElectionProtocolImpl.java \\
Pour implémenter les messages Probe, nous avons décidé d'utiliser une liste supplémentaire pour stocker les noeuds en vie : ceux dont on a reçu un message Probe depuis le dernier delta. On vide cette liste tous les delta temps.
A chaque timer delta, on va faire la différence entre la liste de voisins et la liste des noeuds en vie. Si un noeud n'est pas dans la liste des noeuds en vie, cela signifie qu'on n'a pas reçu de message Probe depuis delta temps au moins, donc il n'est plus considéré comme un voisin.
On ajoute ensuite dans la file d'événements un événement différent des messages pour se \enquote{réveiller} au bout de delta temps.


\subsection{q4}


\subsection{q5}

\lipsum[2] % Dummy text

\begin{align} 
\begin{split}
(x+y)^3 	&= (x+y)^2(x+y)\\
&=(x^2+2xy+y^2)(x+y)\\
&=(x^3+2x^2y+xy^2) + (x^2y+2xy^2+y^3)\\
&=x^3+3x^2y+3xy^2+y^3
\end{split}					
\end{align}

Phasellus viverra nulla ut metus varius laoreet. Quisque rutrum. Aenean imperdiet. Etiam ultricies nisi vel augue. Curabitur ullamcorper ultricies

%------------------------------------------------

\subsection{Heading on level 2 (subsection)}

Lorem ipsum dolor sit amet, consectetuer adipiscing elit. 
\begin{align}
A = 
\begin{bmatrix}
A_{11} & A_{21} \\
A_{21} & A_{22}
\end{bmatrix}
\end{align}
Aenean commodo ligula eget dolor. Aenean massa. Cum sociis natoque penatibus et magnis dis parturient montes, nascetur ridiculus mus. Donec quam felis, ultricies nec, pellentesque eu, pretium quis, sem.

%------------------------------------------------

\subsubsection{Heading on level 3 (subsubsection)}

\lipsum[3] % Dummy text

\paragraph{Heading on level 4 (paragraph)}

\lipsum[6] % Dummy text

%----------------------------------------------------------------------------------------
%	PROBLEM 2
%----------------------------------------------------------------------------------------

\section{Lists}

%------------------------------------------------

\subsection{Example of list (3*itemize)}
\begin{itemize}
	\item First item in a list 
		\begin{itemize}
		\item First item in a list 
			\begin{itemize}
			\item First item in a list 
			\item Second item in a list 
			\end{itemize}
		\item Second item in a list 
		\end{itemize}
	\item Second item in a list 
\end{itemize}

%------------------------------------------------

\subsection{Example of list (enumerate)}
\begin{enumerate}
\item First item in a list 
\item Second item in a list 
\item Third item in a list
\end{enumerate}

%----------------------------------------------------------------------------------------

\end{document}
